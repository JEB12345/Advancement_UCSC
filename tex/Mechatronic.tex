\chapter{Mechatronic Design}

An ideal tensegrity system, either robotic or static, is a collection of rigid compressive elements suspended within a network of tensioned cables. 
For a robotic tensegrities without a payload, the actuation and supporting electronics would be logically designed into the compressive elements. 
Making each compressive element identical will also make the over all design and support of the entire tensegrity robot simplier. 
For SUPERball, each compressive element would be comprised of three parts: two identical end caps and a piece of tube stock. 
The end cap was designed 

\section{Mechanical}
Write about the mechanical design of SUPERball. Make sure to give credit where credit is due.

B.  Mechanical Design
SUPERball  is  an  icosahedron  tensegrity  structure  com-
prised  of  12  motors  at  the  end  of  the  robot’s  6  rods.  Each
rod  is  comprised  of  three  main  elements,  2  modular  end
cap  assemblies  containing  all  the  mechanical  and  electrical
systems   and   a   connecting   aluminum   tube   as   a   support
structure. The main structural elements of the end caps were
kept  simple  and  in  sections  to  enable  each  end  cap  to  be
modular  as  well  as  self  contained  so  that  the  end  cap  may
be removed from the connecting rod as one whole unit. The
end  caps  are  held  onto  the  connecting  rods  by  a  simple
tube  collar  for  easy  removal.  There  are  5  sections  to  the
modular end cap which are, a spring holder, battery holder,
motor and electronics element, cable actuation section, and a
ground contact section. These sections as they are designed
for SUPERball are shown in Fig. 4. Each of these 5 sections
can  be  removed  from  the  rod  as  a  full  sub-assembly  and
replaced with a new component, increasing the versatility of
each rod.
A  lesson  learned  from  ReCTeR  was  that  externally  ex-
posed springs are not ideal for a robotic system. The exposed
springs get caught on objects and the assumption of massless
cables  can  no  longer  be  applied.  On  the  modular  end  cap
for SUPERball, an enclosed compression spring system was
developed  to  alleviate  these  issues.  Compression  springs
were chosen so that during any unknown impact, the springs
would not plastically deform. For SUPERball, a spring with
a spring constant of
613
N=m
is attached to a passive cable
element  and  a
2850
N=m
spring  is  attached  to  an  actuated
cable.  The  passive  spring  has  a  much  higher  compressive
range to allow for pretension to be instated into the passive
springs.  A  working  prototype  of  our  spring  holder  system
can be seen in Fig. 5.

\section{Electrical}
Write about the electronic board used on SUPERball.\\

SUPERball  was  developed  with  distributed  controls  in
mind.  Each  rod  end  cap  houses  two  control  boards,  one
for  motor  driving  and  one  for  handling  sensing  and  com-
munications. Each board hosts a Microchip dsPIC33EP. The
motor driver is a BLDC/PMSM driver board capable of block
commutation and sensorless sinusoidal control. Each sensor
board is equipped with an ADC
(24bit Analog AD7193)
and
9 DOF IMU data
(MPU6000 and MAG3110)
.
Two custom force sensors were developed for the SUPER-
ball, a reaction torque sensor and a compression force sensor.
Fig. 6 shows the reaction torque sensor. It is a symmetrical
four  arm  cross  design  with  the  half  bridge  located  in  the
center of each arm. This sensor, along with the compression
sensors and current sensors allow us to implement high level
control schemes such as impedance control in which the full
state of the mechanical and electrical system must be known.

- Untethered electronics
- 

To this end, we tried to develop electronics which where extensible for various communications
Another driving parameter was the ability to drive the 100W BLDC Maxon motors chosen from the initial iterative design.
These two main design criteria governed our prototype to implement four separate electronic boards per end cap.
Three boards are custom designed PCBs and the fourth is an ARM based computer called Beagle Bone Black.
Each custom PCB is designed for very different purposes: A board to condition sensor data and run real-time control loops, a board to condition and distribute a 5.5V electronic power rail and a 24V motor power rail, and a board to control the 100W BLDC motor. 
The boards are simply named by their main purpose, thus Sensor, Power, and Motor board, respectively.

\subsubsection{Sensor Board}
The sensor board was originally designed as the main processing unit on an end cap for SUPERbal. 
However, the design and building process has lead to the coupling of the sensor board with a Beagle Bone Black.
These two versions are dubbed v1 and v2, respectively.
I will first write about v1 of the sensor board, then follow up on how v2 was changed.\\

The main processing unit on each sensor board is Microchip's dsPIC33EP128GP506, a 16 bit microcomputer running at 140MHz. 
Besides the designer's familiarity with this family of microcomputers, the dsPIC33E chips feature multiple Universal Asynchronous Receiver/Transmitter (UART), Serial Peripheral Interface (SPI), and Inter-Integrated Circuit (I2C) communication modules. 
The dsPIC33E also features an ECAN modules with 2.0B support, a 12 bit Analog to Digital Converter, and four Direct Memory Access (DMA) channels. 
All these moudles coupled with almost complete pin to peripheral pin remapping made the microcomputer a solid choice for our sensor board. 

\section{Communication and Data Flow}



