\chapter{Introduction}

As part of our research for the NASA Innovative Advanced
Concepts  (NIAC)  program,  we  are  developing  the  SUPER-
ball (Spherical Underactuated Planetary Exploration Robot),
which is a compliant icosahedron tensegrity robot designed
for   planetary   landing   and   exploration.   Tensegrity   robots
are  soft  machines  which  are  uniquely  able  to  compliantly
absorb  forces  and  interact  with  unstructured  environments.
However, instead of engineering a single new robot, we have
chosen  to  develop  a  fundamentally  reusable  component  for
tensegrity  robots  by  creating  a  modular  robotic  tensegrity
strut which contains an integrated system of power, sensing,
actuation, and communications. The purpose is to enable the
exploration of the wide range of possible tensegrity robotic
morphologies  by  simply  combining  the  robotic  struts  into
new systems.

It is possible to design free-standing structures by arrang-
ing  axially  loaded  compression  elements  in  a  well  crafted
network of tensional elements. Such an arrangement is called
a tensegrity structure (tensile integrity). Each element of the
structure  experiences  either  pure  axial  compression  or  pure
tension [1][2]. The absence of bending or shear forces allows
for highly efficient use of materials, resulting in lightweight,
yet robust systems.
Because the struts are not directly connected, tensegrities
have  the  unique  property  that  externally  applied  forces  dis-
tribute  through  the  structure  via  multiple  load  paths.  This
creates  a  soft  structure,  for  a  soft  robot,  out  of  inherently
rigid  materials.  Since  there  are  no  rigid  connections  within
the structure, there are also no lever arms to magnify forces.
The  result  is  a  global  level  of  robustness  and  tolerance  to
forces applied from any direction.
This  makes  tensegrity  robots  inherently  compliant  and
extremely well suited for physical interactions with complex
and  poorly  modeled  natural  environments.  Active  motion
in  tensegrity  robots  can  be  performed  by  changing  cable
lengths in parallel, enabling the use of many small actuators
that  work  together,  rather  than  individual  heavy  actuators
which   work   in   series.   There   are   also   many   indications
that tensegrity properties are prevalent throughout biological
systems,  and  the  morphology  of  the  SUPERball  that  we
are  studying,  especially  when  carrying  a  payload,  ends  up
bearing  a  striking  resemblance  to  the  nucleated  tensegrity
model of cell structure.

Because  of  the  limited  research  into  actuated  tensegrity
robotics,   many   design   aspects   have   yet   to   be   carefully
studied. To date, the majority of constructed tensegrity robots
have  been  simple  prototypes  using  servo  motors,  limited
sensing,  and  are  often  tethered  for  power  and  control  [5].
Others have had fewer limbs than the SUPER ball, or have
been  secured  to  the  ground  as  opposed  to  free-standing
[6][7]. Some related approaches utilize tensegrity as part of a
larger, more complicated system, but not as the primary loco-
motion method [8]. Others have created designs that do not
use direct cable actuation, as in the SUPER ball, but instead
have  more  limited  forms  of  locomotion  through  vibration
[9][10]. Finally, the most similar designs to the SUPER ball
have not been engineered to specific design requirements nor
have  the  advanced  sensing  framework  needed  for  controls
testing [11]

The  high  strength-to-weight  ratio  of  tensegrity  structures
is very attractive due to the impact of mass on mission launch
costs.  Large  tensegrity  structures  have  been  shown  to  be
deployable from small compact configurations which enable
them to fit into space constrained launch fairings. While the
above qualities have inspired studies of deployable antennae
and  other  large  space  structures  [12],  it  is  in  the  realm
of  planetary  exploration  that  we  see  the  most  significant
role  for  many  of  the  unique  force  distribution  qualities  of
tensegrity  robots.  A  recent  NIAC  project  [13]  specifically
studies  landing  and  surface  mobility  of  tensegrities,  ex-
ploiting  the  controllable  compliance  and  force  distribution
properties which make for reliable and robust environmental
interactions.
The   main   goal   is   to   develop   tensegrity   probes   with
an  actively  controllable  tensile  network  to  enable  compact
stowage  for  launch,  followed  by  deployment  in  preparation
for  landing.  Due  to  their  natural  compliance  and  structural
force  distribution  properties,  tensegrity  probes  can  safely
absorb significant impact forces, enabling high speed Entry,
Descent, and Landing (EDL) scenarios where the probe itself
acts  much  like  an  airbag.  However,  unlike  an  airbag  which
must  be  discarded  after  a  single  use,  the  tensegrity  probe
can  actively  control  its  shape  to  provide  compliant  rolling
mobility  while  still  maintaining  the  ability  to  safely  absorb
impact  shocks  that  might  occur  during  exploration.  This
combination  of  functions  from  a  single  structure  enables
compact and lightweight planetary exploration missions with
the capabilities of traditional wheeled rovers, but with a mass
and cost similar or less than a stationary probe.

Therefore, a large fraction of the overall weight (as mea-
sured  at  atmospheric  entry)  of  a  tensegrity  mission  can  be
used  for  the  scientific  payload  due  to  the  dual  use  of  the
structure  as  a  lander  and  a  rover.  This  allows  for  cheaper
missions  and  enable  new  forms  of  surface  exploration  that
utilize the natural tolerance to impacts of tensegrities [14].

Buckminster Fuller [1] and the artist Kenneth Snelson [2]
initially  explored  tensegrity  structures  in  the  1960s.  Until
the  mid-1990s  the  majority  of  tensegrity  related  research
was  concerned  with  form-finding  [15]  and  design  analysis
of  static  structure  [16][17].  More  recently,  active  control
efforts  for  tensegrities  began  to  emerge  [18],  as  well  as
descriptions  of  the  dynamics  of  tensegrity  structures  taking
the connectivity pattern into account [17].
The tensegrity principle allows for compliance and multi-
path load distribution, which is ideal for physical interaction
with  the  environment.  However,  these  aspects  also  present
significant  challenges  to  traditional  control  approaches.  A
recent  review  [19]  shows  that  there  are  still  many  open
problems in actively controlling tensegrities, especially when
interacting  with  an  environment  during  locomotion  or  ma-
nipulation  tasks.  Though  work  has  been  done  to  control  a
tensegrity  to  change  into  a  specified  shape  [20],  practical
determination  of  the  desired  shape  itself  is  an  ongoing
challenge.  Recently,  locomotion  of  icosahedral  tensegrity
robots  through  body  deformation  was  demonstrated  [21].
Other work has addressed collision between rigid tensegrity
elements during control generation [22][23].
The  approach  taken  by  the  NASA  Dynamic  Tensegrity
Robotics  Lab  builds  on  this  by  developing  body  defor-
mation  control  algorithms  based  on  central  pattern  gen-
erators  [24][25],  distributed  learning,  reservoir  computing,
and  genetic  algorithms  [26],  instead  of  traditional  linear
and  nonlinear  systems  approaches.  To  date,  our  approach
has  shown  promising  results  at  productively  harnessing  the
potential  of  complex,  compliant,  and  nonlinear  tensegrity
structures.

\section{Motivation}

Though there is much prior work in a variety of theoretical
areas for tensegrities, engineering knowledge of constructing
practical  tensegrity  robots  is  limited.  Since  a  staggering
variety  of  different  tensegrity  structures  can  be  constructed
from  collections  of  simple  sticks  and  strings  (for  example,
see the TensegriToy modeling kit), we have made it a priority
to develop self-contained robotic tensegrity struts which can
be  used  to  explore  and  build  a  wide  range  of  tensegrity
robots  simply  by  combining  them  into  novel  structures.
Our  designs  are  driven  by  experimental  results  obtained
from  a  previous  prototype,  ReCTeR  (Reservoir  Compliant
Tensegrity Robot) in combination with simulation results of
our validated tensegrity simulator NTRT (NASA Tensegrity
Robotics Toolkit) [27][28]

\section{Application}


\section{Goal}

In  order  to  develop  SUPERball  from  ReCTeR’s  design
limitations as well as our lab’s need for rapid experimentation
of  various  tensegrity  configurations  and  morphologies,  we
came  up  with  a  modular  tensegrity  platform  to  research
large  scale  robotic  tasks;  e.g.  a  tensegrity  planetary  probe
to explore Saturn’s moon Titan.
A.  Design Requirements
Our lab obtained design requirements through an iterative
approach  involving  NTRT  and  ReCTeR.  As  we  recently
validated  our  NTRT  simulator  by  experimental  validation
with  ReCTeR  [28],  we  can  now  quickly  evaluate  various
tensegrity  configurations  in  simulation  to  find  optimal  me-
chanical  design  goals.  Next  to  the  NTRT  solver,  we  also
incorporated  results  obtained  with  our  (open  source)  Euler
Lagrange solver based on Skelton’s work [17] and measure-
ments on ReCTeR.
The design requirements obtained from the NTRT simula-
tions are given in Table I. We are confident that a tensegrity
robot achieving the following conditions will be capable of
dynamic locomotion, as shown by our evaluation of control
policies in Section V.

