\chapter{Literature Review}

\section{Tensegrity Structures}
It is possible to design free-standing structures with axially loaded compression elements in a well crafted network of tensional elements.
Such an arrangement is called a tensegrity structure (tensile integrity). 
Each element of the structure experiences either pure axial compression or pure tension \cite{BuckminsterFuller1975}\cite{Snelson1965}.
The absence of bending or shear forces allows for highly efficient use of materials, 
resulting in lightweight, yet robust systems.

Because the struts are not directly connected, 
tensegrities have the unique property that externally applied forces distribute through the structure via multiple load paths. 
This creates a soft structure, for a soft robot, out of inherently rigid materials.
Since there are no rigid connections within the structure, there are also no lever arms to magnify forces. 
The result is a global level of robustness and tolerance to forces applied from any direction.

%Tensegrity structures are composed of axially loaded compression elements encompassed within a network of tensional elements, and thus each 
%element experiences either pure linear compression or pure tension.  As a result, individual elements can be extremely lightweight as there are no 
%bending or shear forces that must be resisted.  Because the struts are not directly connected, a unique property of tensegrity structures is how externally applied forces distribute through the structure 
%via multiple load paths, creating a system level robustness and tolerance to forces applied from any direction.  
%Because there are no rigid connections within the structure, there are also no lever arms to magnify forces.  Instead, all experienced forces act linearly on each structural element.  Combined with the ability to diffuse forces globally, 
This makes tensegrity robots inherently compliant and extremely well suited for physical interactions with complex and poorly modeled natural environments.  
Active motion in tensegrity robots can be performed by changing cable lengths in parallel, 
enabling the use of many small actuators that work together, rather than individual heavy actuators which work in series.  
There are also many indications that tensegrity properties are prevalent throughout biological systems, 
and the morphology of the SUPERball that we are studying, 
especially when carrying a payload, 
ends up bearing a striking resemblance to the nucleated tensegrity model of cell structure~\cite{Wang2009}\cite{Wang2001}.

\section{Prior Work in Tensegrity Robotics Design}
An important advantage of tensegrity structures with respect to general pin-jointed structures is their increased mass-efficiency due to a high fraction of tensile members.
Tensile members are generally more mass-efficient as they do not need to resist buckling.
A further advantage from a robotics perspective is that forces diffuse in a tensegrity.
There are no lever arms and torques do not accumulate at the joints as in a classic serial manipulator.
Forces distribute through multiple load paths, thus increasing robustness and tolerance to mechanical failure.

%advantages
%applications
The static properties of tensegrities have been thoroughly studied and some basic analysis is discussed in section \ref{modeling}.
On the other hand, few examples are known of truly dynamic motion of these structures.
Early examples of kinematic motion include the work at EPFL's IMAC laboratory~\cite{Fest2004}.
Skelton and Sultan introduced algorithms for the positioning of tensegrity based telescopes and the dynamic control of a tensegrity flight simulator platform~\cite{sultan2000tensegrity}.
Although there were some early efforts at MIT's CSAIL lab, it wasn't until the work of Paul and Lipson at Cornell University that the concept of tensegrity robotics became widespread~\cite{Paul2006a}.
Paul and Lipson were the first to study the properties of dynamic tensegrity structures in hardware and simulation.
A few years later Fivat and Lipson designed the IcoTens, a small actuated tensegrity icosahedron robot, but did not publish results.
In recent years, the BIER lab at the University of Virginia has been studying Central Pattern Generator based control for tensegrity based fish tails,
which is closely related to the control architectures proposed for SUPERball~\cite{Caluwaerts2013rsif,Bliss2012}.
Mirats-Tur has presented design and controls work on various other tensegrity morphologies that have been tethered or fixed to the ground~\cite{GraellsRovira2009,miratstur2011athree-dof}.
At Union College, Rieffel and colleagues are following an interesting line of work by considering vibration based actuation for small tensegrities~\cite{khazanov2014developing}.
Related work was presented by B\"ohm and Zimmermann, who demonstrated controlled locomotion of vibration driven tensegrity robots with a single actuator~\cite{bohm2013vibration}.
Finally, Shibata, Hirai and colleagues have developed pneumatically actuated rolling tensegrity structures~\cite{Shibata2009}. 
%goal

Building upon these works, the \SB{} project seeks to push forward the tensegrity robotics field and develop truly untethered, highly dynamic and compliant robots exploiting the aforementioned advantages.

\section{Tensegrity Robotics for Space Exploration}
%NASA is supporting research into tensegrity robotics to create robots with many of the same qualities that benefit biological systems.  
The high strength-to-weight ratio of tensegrity structures is very attractive due to the impact of mass on mission launch costs. 
Large tensegrity structures have been shown to be deployable from small compact configurations which enable them to fit into space constrained launch fairings.   
While the above qualities have inspired studies of deployable antennae and other large space structures~\cite{Tibert2002}, 
it is in the realm of planetary exploration that we see the most significant role for many of the unique force distribution qualities of tensegrity robots.  
The NIAC project currently funding this research~\cite{NIACfinalreport} specifically studies landing and surface mobility of tensegrities,
exploiting the controllable compliance and force distribution properties which make for reliable and robust environmental interactions.  

The main goal is to develop tensegrity probes with an actively controllable tensile network
 to enable compact stowage for launch, followed by deployment in preparation for landing. 
Due to their natural compliance and 
structural force distribution properties, tensegrity probes can safely absorb 
significant impact forces, enabling high speed Entry, Descent, and Landing 
(EDL) scenarios where the probe itself acts much like an airbag.  However, 
unlike an airbag which must be discarded after a single use, the tensegrity 
probe can actively control its shape to provide compliant rolling mobility 
while still maintaining the ability to safely absorb impact shocks that might 
occur during exploration.  This combination of functions from a single 
structure enables compact and lightweight planetary exploration missions 
with the capabilities of traditional wheeled rovers, but with a mass and 
cost similar or less than a stationary probe.   

Therefore, a large fraction of the overall weight (as measured at atmospheric entry) of a tensegrity mission can be used for the scientific payload 
due to the dual use of the structure as a lander and a rover. 
This allows for cheaper missions and enable new forms of surface exploration that utilize the natural tolerance to impacts of tensegrities~\cite{Vytas_IPPW_2013}.

% Because  of  the  limited  research  into  actuated  tensegrity
% robotics,   many   design   aspects   have   yet   to   be   carefully
% studied. To date, the majority of constructed tensegrity robots
% have  been  simple  prototypes  using  servo  motors,  limited
% sensing,  and  are  often  tethered  for  power  and  control  [5].
% Others have had fewer limbs than the SUPER ball, or have
% been  secured  to  the  ground  as  opposed  to  free-standing
% [6][7]. Some related approaches utilize tensegrity as part of a
% larger, more complicated system, but not as the primary loco-
% motion method [8]. Others have created designs that do not
% use direct cable actuation, as in the SUPER ball, but instead
% have  more  limited  forms  of  locomotion  through  vibration
% [9][10]. Finally, the most similar designs to the SUPER ball
% have not been engineered to specific design requirements nor
% have  the  advanced  sensing  framework  needed  for  controls
% testing [11]
%
% The  high  strength-to-weight  ratio  of  tensegrity  structures
% is very attractive due to the impact of mass on mission launch
% costs.  Large  tensegrity  structures  have  been  shown  to  be
% deployable from small compact configurations which enable
% them to fit into space constrained launch fairings. While the
% above qualities have inspired studies of deployable antennae
% and  other  large  space  structures  [12],  it  is  in  the  realm
% of  planetary  exploration  that  we  see  the  most  significant
% role  for  many  of  the  unique  force  distribution  qualities  of
% tensegrity  robots.  A  recent  NIAC  project  [13]  specifically
% studies  landing  and  surface  mobility  of  tensegrities,  ex-
% ploiting  the  controllable  compliance  and  force  distribution
% properties which make for reliable and robust environmental
% interactions.
% The   main   goal   is   to   develop   tensegrity   probes   with
% an  actively  controllable  tensile  network  to  enable  compact
% stowage  for  launch,  followed  by  deployment  in  preparation
% for  landing.  Due  to  their  natural  compliance  and  structural
% force  distribution  properties,  tensegrity  probes  can  safely
% absorb significant impact forces, enabling high speed Entry,
% Descent, and Landing (EDL) scenarios where the probe itself
% acts  much  like  an  airbag.  However,  unlike  an  airbag  which
% must  be  discarded  after  a  single  use,  the  tensegrity  probe
% can  actively  control  its  shape  to  provide  compliant  rolling
% mobility  while  still  maintaining  the  ability  to  safely  absorb
% impact  shocks  that  might  occur  during  exploration.  This
% combination  of  functions  from  a  single  structure  enables
% compact and lightweight planetary exploration missions with
% the capabilities of traditional wheeled rovers, but with a mass
% and cost similar or less than a stationary probe.
%
% Therefore, a large fraction of the overall weight (as mea-
% sured  at  atmospheric  entry)  of  a  tensegrity  mission  can  be
% used  for  the  scientific  payload  due  to  the  dual  use  of  the
% structure  as  a  lander  and  a  rover.  This  allows  for  cheaper
% missions  and  enable  new  forms  of  surface  exploration  that
% utilize the natural tolerance to impacts of tensegrities [14].
%
% Buckminster Fuller [1] and the artist Kenneth Snelson [2]
% initially  explored  tensegrity  structures  in  the  1960s.  Until
% the  mid-1990s  the  majority  of  tensegrity  related  research
% was  concerned  with  form-finding  [15]  and  design  analysis
% of  static  structure  [16][17].  More  recently,  active  control
% efforts  for  tensegrities  began  to  emerge  [18],  as  well  as
% descriptions  of  the  dynamics  of  tensegrity  structures  taking
% the connectivity pattern into account [17].
% The tensegrity principle allows for compliance and multi-
% path load distribution, which is ideal for physical interaction
% with  the  environment.  However,  these  aspects  also  present
% significant  challenges  to  traditional  control  approaches.  A
% recent  review  [19]  shows  that  there  are  still  many  open
% problems in actively controlling tensegrities, especially when
% interacting  with  an  environment  during  locomotion  or  ma-
% nipulation  tasks.  Though  work  has  been  done  to  control  a
% tensegrity  to  change  into  a  specified  shape  [20],  practical
% determination  of  the  desired  shape  itself  is  an  ongoing
% challenge.  Recently,  locomotion  of  icosahedral  tensegrity
% robots  through  body  deformation  was  demonstrated  [21].
% Other work has addressed collision between rigid tensegrity
% elements during control generation [22][23].
% The  approach  taken  by  the  NASA  Dynamic  Tensegrity
% Robotics  Lab  builds  on  this  by  developing  body  defor-
% mation  control  algorithms  based  on  central  pattern  gen-
% erators  [24][25],  distributed  learning,  reservoir  computing,
% and  genetic  algorithms  [26],  instead  of  traditional  linear
% and  nonlinear  systems  approaches.  To  date,  our  approach
% has  shown  promising  results  at  productively  harnessing  the
% potential  of  complex,  compliant,  and  nonlinear  tensegrity
% structures.

% \section{Motivation an Goal}
