\chapter{Conclusion and Future Work}
\label{conclusion}

\color{red}
\section{Current Prototype's Limitations and Possible Design Changes}

\section{Utilizing Locomotion Controllers to Enable High Level Trajectory Planning}


\color{blue}
TODO: Most of the text below will be moved to other sections or deleted 

\section{\SB{} Current State}
At this point, the first goal proposed in section \ref{goal} has been achieved.
\SB{} is composed of twelve Modular Tensegrity Robots (MTR) attached as the ends of 6 rods connected in a icosahedron geometry, and may be seen in figure \ref{fig:SB}.
Preliminary testing of the system, presented in the following sections, has shown many of the basic functions and features that will enable goals two and three.
All data collected in this section was collected through the wireless ROS network with no extra sensors or equipement apart from what was designed into the system, explained in chapter \ref{design}. 

\section{Future Work}

%\subsection{Proposed Controls Methods}

\subsection{Open Loop Locomotion Control}
\label{open_loop}
For clarity, open loop used here is open in regards to the locomotion system's ability to change robotic motor inputs based on environmental sensing.
Work presented in~\cite{iscen2014flop} shows through simulation that a tensegrity system like \SB{} can achieve a rolling gait by deforming the triangle currently in contact with the ground.
Though \SB{} is not fully actuated (all 24 external cables are attached to motors), a derivative of this work may be able to be applied to \SB{}.
Leveraging the experimental results from section \ref{basic_locomotion}, \SB{} can achieve open loop locomotion quite easily with the addition of detecting which face of the robot is on the ground.
To achieve this ground detection, I propose to use the IMU modules on each sensor board to detect earth's gravity field and/or ground contacts when a rod contacts the ground.
Using a basic machine learning technique, like k-nearest neighbor, may enable successful classification of where the ground is in relation to the robot.

\subsection{Closed Loop Locomotion Control}
There has been preliminary results done by~\cite{burms2015online} which demonstrates a tensegrity robot sensing different enviromental terrains.
This shows promise that a tensegrity robot may sense changes in terrain without the need for extra sensors.
If a similar technique can be achieved on \SB{} in a real-time manor, then the open loop gait pattern used from \ref{open_loop} can be altered to better locomote over the sensed terrain.
This new locomotion gait may either be hand tuned parameters or learned behavior.

