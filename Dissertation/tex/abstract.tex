Presented in this work are the concepts to build, sense and control a completely untethered tensegrity robotic system called \SB{} (Spherical Underactuated Planetary Exploration Robot), which is a compliant icosahedron tensegrity robot designed to enable research into tensegrity robots for planetary landing and exploration as part of a NASA funded program.
Tensegrity robots are structurally compliant machines, uniquely able to absorb forces and interact with unstructured environments through the use of multiple rigid bodies stabilized by a network of cables.
However, instead of engineering a single new robot, a fundamentally reusable component for tensegrity robots was developed by creating a modular tensegrity robotic strut which contains an integrated system of power, sensing, actuation, and communications.
\SB{} utilizes six of these modular struts, making the \SB{} system analogous to a swarm of 6 individual robots, mutually constrained by a cable network.

Since \SB{} is intended for use on planetary surfaces without the support of GPS, state estimation and control policies only utilize the sensors on board the robotic system.
When external sensors are used, they must be able to account for imprecise placement and automatic calibration.
Also, dynamic tensegrity systems do not exhibit continuous dynamics due to nonlinear cable conditions and interactions with the environment, thus non-traditional control development methods are implemented.
In this work, control polices are developed using Monte Carlo, evolutionary algorithms, and advanced supervised learning through Guided Policy Search.
Each system is evaluated in simulation, while state estimation and the Guided Policy Search method are additionally evaluated on the physical \SB{} robotic system.