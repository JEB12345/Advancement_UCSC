Early tensegrity research was mostly focused on modeling the statics
~\cite{2003Tensegrity:-Str, Juan2008, Arsenault:2008bh} and dynamics
of a structure, so as to provide effective equations of motion
~\cite{Kanchanasaratool2002Motion-Control-, De-Oliveira:2006ys,
skelton_tensegrity_2009}.  In specific cases, such as state
estimation, modeling tensegrities as constrained mass-spring nets
allows for highly efficient and sufficiently accurate
implementations~\cite{caluwaerts2016esitmation}.  The mass-spring approach
has also proven valuable in more theoretical studies of morphological
computation~\cite{Hauser}. Nevertheless, there is a trade off between
computational efficiency and high dynamic model fidelity during
environment interaction modeling.  Table~\ref{tbl:resources}
summarizes many of these contributions that can impact tensegrity
control.

%But the dynamics of tensegrities is an active research area being
%approached by a variety of viewpoints~\cite{skelton_tensegrity_2009}.


Many of the tensegrity hardware robots are tendon-driven or use
pneumatic actuators, which are typically a burden to accurately model
analytically. A learned model might increase the computational
efficiency when trying to represent real-world hardware in changing
environments.  Section~\ref{sec:machine_learning} discusses the option of
using learned models as a proxy to simulation or involved analytical
models.

%Given models of tensegrity structures, many interesting problems
%regarding tensegrity structures correspond to their
%control \cite{Tur:2009fu}.

Given a dynamic model, it is possible to control a tensegrity
structure along static equilibrium
manifolds~\cite{Skelton1997Controllable-Te}.  Alternatively, feedback
linearization control laws~\cite{Aldrich2003Control-Synthes} or
Lyapunov-based controllers for 3D dynamic
models~\cite{Wroldsen2006A-Discussion-on} have also been
developed. Frequently, these approaches do not account for
self-collisions or environmental contact dynamics, limiting their
real-world applicability.  Planning processes for a real-world
tensegrity structure need to utilize modeling and simulation tools
that take collisions into account.

%This is an important component of simulation tools, which can help to
%study controlled actuation, as these collisions generate the forces
%to propel the system~\cite{Rovira2009Control-and-Sim,
%Wittmeier2011CALIPER:-A-univ}.

Many efforts focus on generating efficient gaits, defined as rhythmic
motions, which lead to nonzero movement of the center of
mass~\cite{McIsaac:2003kl}.  Given the high-dimensional nature of the
search space, genetic algorithms are frequently applied to achieve
forward locomotion gaits~\cite{Paul2006a}. Evolutionary algorithms
have been used for generating irregular locomotion and civil
engineering structures~\cite{Rieffel2009Automated-Disco,
veuve2015deployment}. Recently, evolutionary methods have been
proposed that utilize a multi-agent learning
approach~\cite{Iscen2013Controlling-Ten}.  Other biologically-inspired
approaches based on Central Pattern Generators (CPGs) have also
been applied to tensegrity-based
systems~\cite{Bliss2013Central-Pattern, MirletzSoftRobotics,
Caluwaerts2013rsif}. 
A recent overview of low-level tensegrity
control approaches is available in the related literature~\cite[Table
2]{Caluwaerts2013rsif}. 
In this work, Monte Carlo~\cite{doucet2001introduction} and evolutionary techniques~\cite{wiegand2001empirical} are used to learn open loop policies for locomotion.
While an Artificial Neural Network (ANN) is trained as a closed loop locomotion controller.
Section~\ref{sec:ann} gives a basic overview of ANNs.

% \begin{figure*}[t]
% \includegraphics[width=0.235\textwidth]{figures/_0_1_extended.jpg}
% \hspace{0.05in}
% \includegraphics[width=0.235\textwidth]{figures/_0_3_extended.jpg}
% \hspace{0.05in}
% \includegraphics[width=0.235\textwidth]{figures/_0_5_extended.jpg}
% \hspace{0.05in}
% \includegraphics[width=0.235\textwidth]{figures/_0_7_extended.jpg}
% \vspace{-.25in}
% \caption{\footnotesize Resulting trajectory for a physically simulated
%   tensegrity robot \cite{SunSpiralSoftware} using an asymptotically, 
%   near-optimal kinodynamic planner
%   \cite{Li2015Sparse-Methods-}.}
%  \vspace{-.2in}
% \label{fig:tens_example}
% \end{figure*}

%such as an experimental robotic
%swimmer \cite{Bliss2013Central-Pattern}, spine-like
%tensegrities~\cite{Tietz2013Tetraspine:-Rob} and the tensegrity
%icosahedron~\cite{Caluwaerts2013rsif}.  

%This line of work tries to take advantage of the fact that forces in tensegrity structures tend to propagate in a distributed way.

%What should the future hold for low-level tensegrity control methods?

The availability of simulation tools has offered researchers the
possibility to develop a wide range of controllers.  Nevertheless,
additional hardware validation results are needed to better support
the claims regarding the efficacy of the developed
solutions~\cite{Mirletz2015, Caluwaerts2013rsif}.  It is crucial to
determine the feasibility of each method in terms of sensing and state
estimation, their aptitude for distributed implementations and the
minimum number of actuators required.

Furthermore, hardware experiments have not typically utilized
fundamental analytical control approaches (e.g.~\cite{sultan2002}),
since they frequently depend on accurate state information, which is
non-trivial to acquire.  Nevertheless, there has been recent progress
on actuation, sensing, and state-estimation methods that are robust to
noisy sensors and environments.  Thus, it may be time to revisit some
of the earlier analytical control techniques. This will allow a
thorough comparison with more modern methods that have reduced sensing
and actuation requirements in simulation and hardware.

%This will serve as a springboard for the
%development of useful local maneuvers for tensegrity robots that can
%be used for long-term mobility purposes in the context of realistic
%terrains and for three-dimensional, large, irregular tensegrity
%structures.

%\komment{Ken: Do we need the above paragraph? Kostas: I personally like it
%because it takes a stance. It is a reasonable recommendation.}

%Low-level controllers can provide primitives that can be used as
%building blocks for global path planning approaches.  For this to be
%plausible, a common parametrization for low-level controllers is
%needed and a method to predict the probability of success or the
%region of attraction of a controller when a low-level controller is
%activated to transition between desired states

%\komment{Ken: I don't like the above sentence, but we need its message in
%the paper Kostas: Since it is going to appear later, perhaps we do not
%need to say it here.}
