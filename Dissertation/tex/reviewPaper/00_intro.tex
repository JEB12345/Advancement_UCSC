Tensegrity structures are examples of soft mechanisms that naturally
perform morphological computation as they can adapt to and
redistribute applied and internally generated forces. Their compliance
and tunable structural stiffness make them desirable robotic
platforms.  In particular, tensegrity robots can safely efficiently
interact with unstructured, real-world environments and tasks that
traditional robots struggle to adapt to. Recently, there has been
progress on the design, control, simulation, and hardware prototyping
of tensegrity robots~\cite{skelton_tensegrity_2009,
Caluwaerts2013rsif}.  Given this progress, they have been shown to be
able to locomote through a variety of terrains~\cite{iscen2014flop,
MirletzSoftRobotics, Caluwaerts2013rsif}.

%This has also resulted in fundamental insights regarding system
%engineering, sensing, actuation, state estimation, and reactive
%low-level control techniques.

In order for the potential of these robots to be fully realized,
progress must also be made on long horizon planning algorithms. Such
methods can enable tensegrity robots to address purposeful navigation
in complex terrains as well as complete sensing and manipulation tasks
that may involve sophisticated deformations.  This work aims to
identify the underlying challenges in developing long-range planning
algorithms for structurally compliant tensegrity robots.  Towards this
objective, this paper surveys existing work on low-level control for
such systems and identifies ways that planning can reason about the
inherent morphological computation and soft mechanism principles of
tensegrities.



