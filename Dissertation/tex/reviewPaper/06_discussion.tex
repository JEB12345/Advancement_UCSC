Tensegrity robots exhibit desirable properties, such as compliance,
force distribution and adaptability to unstructured spaces, which make
them good candidates for critical applications, such as the
exploration of natural and extraterrestrial environments. The same
characteristics, however, also complicate the process of long-horizon
planning for such platforms, since it becomes harder to predict their
motion.

This paper highlights recent progress in tensegrity robotics and
developments in control, simulation, planning and learning, which
bring the promise of formulating efficient solutions, despite the
underlying challenges.  Fig.~\ref{fig:overview} provides an overview
of the related research topics, which at a high-level can be split
between model-based and model-free methodologies. Many of the
promising research directions lie at the intersection of these two
domains and involve the integration of a) high-level planning
strategies, which effectively and verifiably explore the state space
of a tensegrity robot, with b) lower-level controllers, which utilize
morphological computation principles to abstract away the complexity
of controlling a high-dimensional, compliant platform.

The integration of the planning solutions discussed here with
methodologies for other related topics, such as state
estimation \cite{Caluwaerts:2015aa}, will enable the effective
operation of tensegrity robots in meaningful applications.
Furthermore, the study of tensegrity robots can have an impact beyond
their specific class of robots, as it can advance the state-of-the-art
in the wider field of soft and compliant robots. Effective solutions
for tensegrity robots will provide insights into how to deal with
high-dimensionality, compliance, non-linear dynamics, uncertainty and
inaccurate models.


 
