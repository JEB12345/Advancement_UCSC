Tensegrities are variable stiffness, structurally compliant systems,
which distribute all forces into pure linear compression or tension
elements. Thus, they physically change their structural expression
based on experienced forces~\cite{skelton_tensegrity_2009}.
Tensegrities were first explored in art and architecture, where they
have been described as \emph{forces made
visible}~\cite{snelson_forces}. They have also been called the
fundamental building structures of life as they closely resemble the
structure of spines \cite{Levin:2002aa} and the cytoskeletons of
single-celled organisms, which are known to move \cite{Ingber:1998fk}.

\begin{figure}[t]
\includegraphics[width = \linewidth,trim={0cm 0 0
0cm},clip]{figures/tensegrity_fig.png}
\vspace{-.3in}
\caption{Tensegrity robots examples: a) NASA SUPERball robot: an
actuated tensegrity icosahedron with 6 compressive and 24 compliant
elements in tension~\cite{sabelhaus2015system}.  b) The DuCTT v2 duct
climbing robot: 2 linked tetrahedral frames connected by 8 actuated
cables act as a compliant 6 DoF joint~\cite{6907473}.}
\vspace{-.2in}
\label{fig:tensPrinciple}
\end{figure}

Tensegrities inherently integrate structural design and
control \cite{skelton_tensegrity_2009}. A tensegrity structure, such
as the model of Fig.~\ref{fig:tensPrinciple}(a), keeps its shape due
to the balance among tensile and compressive forces. This allows
individual elements of the structure to be lightweight, as they do not
independently need to resist bending or shear forces. Instead, such
forces are distributed across the entire structure and they do not
magnify around joints or other points of failure.  Furthermore, by
equally changing the tension in the cables, the robot can change its
structural stiffness without changing shape, allowing it to adapt to
contact dynamics.  Thus, tensegrity structures are ideally suited for
unstructured and dynamic environments, where contact forces cannot be
easily predicted and traditional mechanisms are not as
adaptive \cite{Zimmermann:2009cr}.  These benefits have motivated the
development of hybrid-soft robots that actively utilize tensegrity
principles, since they can safely operate in real world
conditions~\cite{Trivedi:2008nx, Albu-Schaffer:2008oq}.

\subsection{Applications}

Given the above properties, tensegrity robots enable a wide range of
applications, which involve complex contact dynamics and require
low-weight, energy efficient, compliant platforms.  They can be
designed with a wide variety of shapes and motion dynamics for
different tasks. Multiple behaviors have already been demonstrated:
crawling~\cite{Paul2006a,MirletzSoftRobotics},
swimming~\cite{Bliss2013Central-Pattern}, rolling
locomotion~\cite{iscen2014flop, kimrobust}, and deployment as
compliant many-DoF joints as in Fig.~\ref{fig:tensPrinciple}(b).

Example uses of tensegrities include search and rescue in disaster
areas, the exploration of natural environments, such as volcanic
surfaces and caves, exploration and cleaning of pipes,
ducts \cite{6907473}, and space-related
applications \cite{Furuya:1992dq}.  Especially in the context of
planetary exploration, a tensegrity robot: i) introduces small weight
overhead relative to a science payload, ii) protects sensitive
instruments by acting as an air-bag, iii) provides surface mobility
and iv) manipulation capabilities.  This implies that future space
exploration missions can be cheaper and consider new ways of
interacting with planetary surfaces that take advantage of the natural
tolerance of tensegrity structures to
impacts \cite{SunSpiral2013Tensegrity-Base}.

\subsection{Hardware Prototypes}

Early prototypes were focused on form finding or basic motion
primitives and had limited to no sensor feedback~\cite{Paul2006a,
Rovira2009Control-and-Sim, miratstur2011athree-dof, Shibata2009,
bohm2013vibration}.  More recently, ReCTeR is an un-tethered
tensegrity prototype that allows for closed-loop
locomotion~\cite{Caluwaerts2013rsif}.  A similar geometry tensegrity
robot is SUPERball, which is a prototype planetary exploration rover
for NASA to study dynamic locomotion and path
planning~\cite{bruce2014design,sabelhaus2015system}.
Figure~\ref{fig:SB} illustrates the capabilities of the SUPERball.
Table~\ref{tbl:resources} provides a more extensive list of tensegrity
robot hardware designs.

\begin{figure}[th]
\vspace{-.1in}
\centering
\includegraphics[width=\columnwidth]{figures/fig2_combined.jpg}
\vspace{-.3in}
\caption{SUPERball tensegrity robot prototype: a) during a test
      in the NASA Ames roverscape. b) active crouching by contracting
      cables (background and cables removed) c) deformation during
      droptest d) rolling from face to
      face.}  
\vspace{-.2in}
\label{fig:SB}
\end{figure}


